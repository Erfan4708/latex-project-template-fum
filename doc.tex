% === Load Template and Preamble ===
\newcommand{\coursename}{برنامه سازی پیشرفته}
\newcommand{\projecttitle}{بازی چهار مهره در یک ردیف}
\newcommand{\teachername}{دکتر محمود امین‌طوسی}
\newcommand{\groupmembers}{نام و نام خانوادگی اعضا}
\newcommand{\termname}{بهار ۱۴۰۴}

\documentclass[a4paper,12pt]{article}

% --- Packages ---
\usepackage{xepersian}
\usepackage{graphicx}
\usepackage{fancyhdr}
\usepackage{float}
\usepackage{caption}
\usepackage{longtable}
\usepackage{amsmath}
\usepackage{setspace}
\usepackage{enumitem}
\usepackage{xcolor}
\usepackage[
	colorlinks=true,
	linkcolor=blue,
	urlcolor=blue,
	citecolor=blue
]{hyperref}

\usepackage[a4paper, margin=1in]{geometry}
\setcounter{secnumdepth}{3}

% --- Fonts ---
\settextfont[Path=src/fonts/,Extension=.ttf]{XB Niloofar}
\setlatintextfont[Scale=1.0]{Times New Roman} 
\setdigitfont[Path=src/fonts/,Extension=.ttf]{XB Niloofar}

% --- Paragraph and Spacing Settings ---
\setlength{\headsep}{1.2cm}
\linespread{1.4}

% --- Header and Footer Style ---
\pagestyle{fancy}
\setlength{\headheight}{25pt}
\fancyhf{}
\fancyhead[L]{\projecttitle}
\fancyhead[R]{پروژه درس \coursename}
\fancyfoot[C]{\thepage}


% --- Custom Command: Image Placement Helper ---
\newcommand{\settingimage}[1]{
	\begin{center}
		\includegraphics[width=0.9\textwidth]{#1}
	\end{center}
}


\begin{document}
	
	% === Cover Page ===
	% --- Cover Page Layout ---

\thispagestyle{empty} % Disable header/footer on the cover page

\centering
	\includegraphics[width=0.2\textwidth]{src/images/university_logo.png} 
\vspace{0.5cm}

{\Large دانشکده علوم ریاضی}\\[0.2cm]
{\Large دانشگاه فردوسی مشهد}\\[1.5cm]

{\LARGE \textbf{پروژه درس \coursename}}\\[0.5cm]
\textbf{\projecttitle}
\vspace{1cm}

{\Large استاد درس: \teachername}\\[1.2cm]

{\Large اعضای گروه:}\\[0.3cm]
{\large \groupmembers}\\[3cm]

\vfill
{\large \termname}

\newpage

	
	\raggedleft
	
	\newpage
	
	\tableofcontents
	
	\newpage
	
	% -----------------------------------------------
	% Type here...
	
	\section{مقدمه}
	در این پروژه به پیاده‌سازی بازی چهار مهره در یک ردیف \lr{(Connect Four)} با استفاده از زبان \lr{C++} پرداخته‌ایم. هدف این بازی قرار دادن چهار مهره‌ی یک‌رنگ در یک ردیف افقی، عمودی یا قطری است.
	
	\section{ساختار کلی پروژه}
	پروژه شامل بخش‌هایی برای مدیریت صفحه بازی، ورود کاربر، بررسی برد و... می‌باشد.
	
	\section{نمونه کد}
	\begin{latin}
		\begin{verbatim}
			bool checkWin(vector<vector<char>>& board, char player) {
					// check rows
					for (int i = 0; i < board.size(); ++i)
							for (int j = 0; j < board[0].size() - 3; ++j)
									if (board[i][j] == player && board[i][j+1] == player &&
											board[i][j+2] == player && board[i][j+3] == player)
											return true;
					return false;
			}
		\end{verbatim}
	\end{latin}
	
	% -----------------------------------------------
	
\end{document}
